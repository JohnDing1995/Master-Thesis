\chapter{Implementation of DevOps Toolchain}
In the chapter we introduce the implementation of our DevOps toolchain which act as the basic environment of our experiments. Then we introduce the experiment that for answering RQ1: \textbf{How could DevOps toolchain make use of current cloud services and how these services improves the DevOps toolchain.} In section 4.1 we introduce the implementation of our testing DevOps toolchain which is according to the DevOps definitions and DevOps practices we introduced at Chapter 2. In section 4.2 we discuss the cloud services selection consideration for the experiment. In section 4.3 we introduce the implementation of our experiments. Section 4.4 focuses on the experiment result.
\section{DevOps Toolchain Implementation}
Our DevOps toolchain is used to conduct experiment that could answering 2 research questions. As we mentions above, it include DevOps elements we introduced at CH2. In this section we will first introduce the composition of our DevOps toolchain, and secondly, which elements of DevOps does each components belongs to.

// Introduce our initial design of Devops toolchain

// Point out the problem, and point out the improvement can be done in the toolchain.

% \section{Cloud Services}
% \label{assumption}
% In this section, we will introduce several could services from CH3 that could be helpful to the DevOps toolchain. 
% //  Using services in AWS as example, Introduces how cloud services could improve. describe on services in one section
% \subsection{Managed Container Services for Distributed Builds} 

% // Describe how AWS Fargate could Help
% \subsection{Serverless computing}
% // Describe how AWS lambda could Help and why do we chose it
% \subsection{...}
