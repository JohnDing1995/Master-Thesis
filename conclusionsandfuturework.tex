\chapter{Conclusions and Future Work}
\label{chp:conclusionsandfuturework}
The changes in software development and businesses has affected teams to aim for more efficient practices and tooling which are they key factors of DevOps. Thus in this master thesis project, I focus on the tools aspect within the DevOps, more specifically, is the DevOps toolchain that helps the software team to implement DevOps practises. On the other hand, the development of cloud computing technologies, especially serverless computing, has brought many changes in the development and deployment of DevOps toolchains. Serverless computing services in the cloud could either act as the runner of automated build and automated testing, monitoring tools or even replace the whole DevOps toolchain with a hosted integrated DevOps toolchain. With the motivation to combined the DevOps toolchain and serverless computing, this master thesis project explores the possibility that serverless computing could improve the DevOps toolchain in term of performance and functionality. 
This is done by, first, implemented a traditional non-integrated toolchain and deploy to AWS, and find what improvement could AWS serverless computing services bring to this toolchain for answering RQ1. Second, implement an integrated toolchain with AWS serverless DevOps tools, and compare with the traditional toolchains, which answer my RQ2. 

In this chapter, I first make conclusions by summarising my findings in chapter 3 and chapter 4. The conclusion is presented to give answer to my research questions in section 6.1. In section 6.2, I discuss the possible future works.
\section{Conclusions}
In this section, I summarise my main findings during the implementation and evaluation as the answer to two research questions.
\paragraph{RQ1:} \textit{How can serverless computing services in Amazon Web Services helps the DevOps toolchain?}
\medskip
\par
I identified there are following serverless computing services that could be improving the DevOps toolchain within AWS.
\par
The first improvement is made by the managed serverless container service (AWS ECS with Fargate), which could work as the build agent which host the build and testing process within the CD pipeline. According to my experiments and analysis, the feature of serverless computing, combined with the beneficial brings by Docker container, could improve the parallel execution performance and save cost. The high performance in parallel pipeline execution could be helpful in several aspects. 

First, it improves the performance of continuous delivery for the microservices project. A microservices could have hundred of services and in the philosophy of microservices, the release of each services a microservices project should be independently \cite{dehghani2018break}. To ensure this, one practices it to use one pipeline for each services. However managing hundred pipeline is not a easy task, and a better practices is to have multiple services share a single pipeline \cite{HowtoSca9:online}. Therefore, when multiple services share the same pipeline, this improvement allows services to be released independently without waiting for each other.
Second, the serverless container service reduces the effort when setting up the build agents. Docker build agent is widely used by different CI tools, and Elastic Container Services has extensive API which could be used for deploying the Docker build agent. Therefore, this benefit is not only limited to the tools that I used in my implementation. My experiments also show that Pay-per-use model makes Fargate cheap to use for hosting the build agent.
\par
Moreover, the serverless function (AWS Lambda) could be used within the monitoring part of the toolchain. The event-driving computing model of serverless functional is perfect for processing with the logs, alarms and realtime performance metrics within monitoring. To a software team that sets up DevOps toolchain, they can use AWS Lambda to extend the monitoring system, such as customizing metrics and sending notification. In addition to these, AWS Lambda can helps software team to set up a AWS-event-driven workflow that triggers Jenkins job when changes happens in AWS services. An example could be enable Jenkins build starts when new data is uploaded to S3 bucket.
\par
The third improvement is from the serverless managed tools (AWS CloudWatch e.t.). Those tools could be used directly as a component in the toolchain. For example, A software team can use CloudWatch directly as the monitoring solution in the toolchain. This type of tools frees the development team from developing and maintaining their monitoring solution. Suppose these tools are provided by the cloud provider that deploys the DevOps toolchain. In that case such as monitoring AWS deployed DevOps toolchain with CloudWatch, the good integration between the tool and the cloud may provide more detailed monitoring to the cloud infrastructure than any self-built third-party tool.
\par
There is still limitation within serverless computing services. One most significant limitation is that the Fargate does not support runs privileged container. This was largely limiting what I operation can I do within the running Docker build agent; for example, I cannot build a Docker image for the project that is being built in the build agent. Moreover, the performance of the build agent runs in the serverless computing service is loIr. This is due to the cold-start problem of serverless computing. In addition, the average time of build stage of the same software project is longer in the serverless build agent (92s) than in EC2 based build agent(65s). This difference shows that the serverless computing engine has loIr performance when it runs the automated build and automated testing within a continuous delivery pipeline.
\paragraph{RQ2:} \textit{How does the newly emerged integrated toolchain in Amazon Web Services compare with the traditional non-integrated toolchain?}
\medskip
\par
The integrated toolchain is a new type of DevOps toolchain that delivered as a single application. It is usually composed of serverless tools from a single vendor (in my case: AWS) but could also integrate with a limited range of third-party tools depends on the vendor. Compared with the non-integrated toolchain, The first advantage for integrated toolchain is it is an out-of-box solution that needs much less time and effort to develop and setup, while many works are needed when building the non-integrated toolchain, namely, tool selection, cloud infrastructure design and management, manual configuration and even software development. In addition, an integrated toolchain provides a better sight on the DevOps toolchain as a whole. This is easier to achieve because the integrated toolchain is delivered as a single application. In my experiment on performance, I find that the integrated toolchain is faster in provisioning the computing resource for continuous delivery. I believe due to the computing resources are also managed by AWS, it easier for AWS to optimize the provisioning process.
\par
I also find that, under the same hardware configuration, it takes the integrated toolchain longer time to run the automated build and automated testing. The gap of the runtime betIen integrated and non-integrated toolchain increased with the number of parallel executing jobs.
my research shows this is because the AWS is imitating the maximum hardware resources that I could use. The restriction shows the first disadvantage of the integrated toolchain -- the development team does not have full control on the underlying hardware. A similar problem with integrated toolchain is that the team also has low visibility on the status of the underlying hardware, which brought some challenges in my implementation. Moreover, my analysis shows that the cost of the integrated toolchain is higher in the non-integrated toolchain. The high cost is because, first, not possible to manly use open-source solution, second, the AWS charges much more on computing resource when using it from it's DevOps toolchain.
Furthermore, the integrated toolchains limit the tool selection, which on the one hand reduce the time needed for tool selection, on the other hand, it limiting the use of some special tools that are not supported by the toolchain.
\par
In my opinion, The AWS integrated toolchain has a higher cost. However, the higher price also means better services and less time and effort to develop and maintain the toolchain. The team also need to consider if they have specific tools they need to use that cannot be integrated with AWS toolchain. In the performance aspect, the delivery speed of AWS toolchain could vary dramatically depends on the service number in the microservices and the project complexity. Generally, it is slower compared with Jenkins build agent with the same hardware consideration. The team need to consider if the speed is a sensitive factor in their business, and if the project needs parallel pipeline execution.
\section{Future Work}
Based on my current implementation and findings, I propose following further works could be done.
\begin{itemize}
    \item Due to the tight schedule in finishing the thesis, my case project is rather simple, and the test cased are rather small.  With the increasing popularity of microservices architecture, it is interesting if I extend the case project to a microservices. Such an extension could better simulate the software project in real-life development.
    \item Current my non-integrated toolchain does not support run Docker within Jenkins build agent. As a result, the Docker build task has to be executed on the master node. The Docker build luckily quite light in my case project. HoIver, if the Docker build supposes to take a long time with a heavier case project, the non-parallel execution of this step could cause a serious delay. A solution is needed for solving this problem.
\end{itemize}

