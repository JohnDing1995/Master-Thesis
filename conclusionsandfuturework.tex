\chapter{Conclusions and Future Work}
\label{chp:conclusionsandfuturework}
The efficiency and the agility of DevOps are driving more and more software development teams to perform the transition to DevOps. This master thesis project is done in Eficode Oy, a Finnish multinational company that one of its main tasks is to help costumers with the DevOps transition. The transition to DevOps needs the team to precede the change in process, culture and tools. In this master thesis project, we focus on the tools aspect within the DevOps, which is the DevOps toolchain that helps the software team to implement DevOps practises. 

The development of cloud computing technologies, especially serverless computing, has brought many changes in the development and deployment of DevOps toolchains. Serverless computing services in cloud could either act as the runner of automated build and automated testing, monitoring tools, or even replace the whole DevOps toolchain with a hosted integrated DevOps toolchain. This master thesis project explore the possibility that the serverless computing could improve the DevOps toolchain in term of performance and functionality. This is done by, first, implemented a traditional non-integrated toolchain and deploy to AWS, and find what improvement could AWS serverless computing services bring to this toolchain for answering RQ1. Second, implement a integrated toolchain with AWS serverless DevOps tools, and compare with the traditional toolchains, which answer our RQ2. 

In this chapter, we first make conclusions by summarise our findings in chapter 3 and chapter 4. The conclusion is presented by give answers to our research questions in section 6.1. In section 6.2, we discuses the possible future work that could be done in the future.
\section{Conclusions}
In this section we summarise our main findings during the implementation and evaluation as the answers to two research questions.
\paragraph{RQ1:} \textit{How can serverless computing services in Amazon Web Services helps the DevOps toolchain?}
\medskip
\par
We identified there are following serverless computing services that could improving the DevOps toolchain within AWS.
\par
The first improvement is done by the managed serverless container service (AWS ECS with Fargate) which could works as the build agent which host the build and testing process within the CD pipeline. According to our experiments and analysis, the feature of serverless computing, combined with the beneficial brings by Docker container, could improve the parallel execution performance and save cost. The high performance in parallel pipeline execution could be helpful in, first improving the performance of continuous delivery for microservices project, second, improving the performance of pipeline of a large origination with the frequent delivery. Furthermore, the serverless container service reduces the effort when set up the environment for running the agents. Docker build agent is widely used by different CI tools, and Elastic Container Services has extensive API which could be used for deploying the Docker build agent, therefore, this benefit is not only limited to the tools that we used in our implementation. Our experiments also shows that, Pay-per-use mode makes Fargate cheap to use for hosting the build agent.
\par
Moreover, the serverless function (AWS Lambda) could be used within the monitoring part of the toolchain. The event-driving computing model of serverless functional is perfect for processing with the logs, alarms and realtime performance metrics within monitoring. In our toolchain we use AWS Lambda for customizing metrics and do auto-scaling upon alarms. In addition to these, AWS Lambda could be helpful to in build notification system that keeps the team aware the status within the pipeline.
\par
The third improvement is from the serverless managed tools (AWS CloudWatch e.t.). Those tools could be used directly as tool in the toolchain. For example, we use CloudWatch as the monitoring solution in the toolchain. This type of tools free the development team from developing their own solution for certain part of the toolchain. If the tools is provided by the cloud provider where the DevOps toolchain is being deployed, the good integration between the tool and cloud infrastructure may allow a more detailed monitoring than any self-build or third party tools.
\par
There are still limitation within serverless computing services. One most significant limitation is the Fargate does not support runs privileged container. This largely limiting what we operation can we do within the running Docker build agent, for example, we cannot build Docker image for the project that is being built in the build agent. Moreover, the performance of build agent runs in the serverless computing service is lower. This is due to the cold-start problem of the serverless computing. In addition, the average time of build stage of the same software project is longer in the serverless build agent (92s) than in EC2 based build agent(65s). This difference shows that serverless computing engine has lower performance when runs the automated build and automated testing within a continuous delivery pipeline.
\paragraph{RQ2:} \textit{How does the newly emerged integrated toolchain in Amazon Web Services compared with the traditional non-integrated toolchain?}
\medskip
\par
The integrated toolchain is a new type of DevOps toolchain that delivered as a single application. It is usually composed by serverless tools from a single vendor (in our case: AWS) but could also integrated with a limited range of third-party tools depends on the vendor. Compared with non-integrated toolchain, The first advantage for integrated toolchain is it is a out-of-box solution that needs much less time and effort to develop and setup, while a lot of works are needed when build the non-integrated toolchain, namely, tool selection, cloud infrastructure design and management, manual configuration and even software development. In addition, an integrated provides a better sight on the DevOps toolchain as a whole, this is easier to achieve because the integrated toolchain is delivered as a single application. In our experiment on performance, we find that the integrated toolchain is faster in provisioning the computing resource for continuous delivery. We believe due to the computing resources are also managed by AWS, it it easier for AWS to optimize the provisioning process.
\par
We also find that, under the same hardware configuration, it takes the integrated toolchain longer time to run the automated build and automated testing. The gap of the runtime between integrated and non-integrated toolchain increased with the number of the parallel executing jobs.
Our research shows this is because the AWS is imitating the maximum hardware resources that we could use. The restriction shows the first disadvantage of the integrated toolchain -- the development team do not has fully control on the underlying hardware. A similar problem with integrated toolchain is that the team also has low visibility on the status of underlying hardware, which brought some challenges in our implementation. Moreover, our analysis shows the cost of integrated toolchain is higher in non+integrated toolchain. The high cost is because, first, not possible to manly use open-source solution, second, the AWS charges much more on computing resource when use it from it's DevOps toolchain.
Furthermore, the integrated toolchains limits the tool selection, which on the one hand reduce the time needed for tool selection, on the other hand, it limiting the use of some special tools that is not supported by the toolchain.
\section{Future Work}
Based on our current implement and findings, we propose following further works could be done.
\begin{itemize}
    \item Due to the tight schedule in finishing the thesis, our case project is rather simple, and the test cased are rather small.  With the increasing popularity of microservices architecture, it could be interesting if the case project could be expended to a microservices which include different type of services. This could better simulate the software project in the real-life development.
    \item Current our non-integrated toolchain does not supporting run Docker within Jenkins build agent. As a result the Docker build task has to be executed on master node. The Docker build is luckily quite light in our case project, however if the Docker build is suppose takes longer time with a heavier case project, the non-parallel execution of this step could caused serious delay. A solution is needed for solving this problem.
    \item The monitoring solution in both solution could be extended, currently the monitoring is only based on the solution from AWS. The extension could be done by integrating CloudWatch with third-party tools, for example Grafana.
\end{itemize}

