\chapter{Conclusions and Future Work}
\label{chp:conclusionsandfuturework}
The changes in software development and businesses have affected teams to aim for more efficient practices and tooling, which are the key factors of DevOps. Thus in this master thesis project, we focus on the tools aspect within the DevOps, more specifically, is the DevOps toolchain that helps the software team to implement DevOps practises. On the other hand, the development of cloud computing technologies, especially serverless computing, has brought many changes in the development and deployment of DevOps toolchains. Serverless computing services in the cloud could either act as the runner of automated build and automated testing, monitoring tools or even replace the whole DevOps toolchain with a hosted integrated DevOps toolchain. With the motivation to combine the DevOps toolchain and serverless computing, this master thesis project explores the possibility that serverless computing could improve the DevOps toolchain in term of performance and functionality. 
This is done by, first, implemented a traditional non-integrated toolchain and deploy to AWS, and find what improvement could AWS serverless computing services bring to this toolchain for answering RQ1. Second, implement an integrated toolchain with AWS serverless DevOps tools, and compare with the traditional toolchains, which answer to our RQ2. 

In this chapter, we first make conclusions by summarising our findings in Chapter 3 and Chapter 4. The conclusion is presented to answer our research questions in Section 6.1. In Section 6.2, we discuss possible future works.
\section{Conclusions}
In this section, we summarise our main findings during the implementation and evaluation as the answer to two research questions.
\paragraph{RQ1:} \textit{How can serverless computing services in Amazon Web Services helps the DevOps toolchain?}
\medskip
\par
Our main conclusion for RQ1 is that serverless computing services can benefit the DevOps toolchain by improving performance, eliminating server management tasks, extend functionalities and reducing the cost of idle time.
We identified that from following aspects, the AWS serverless computing services could improve the DevOps toolchain.
\par
The first improvement is made by the managed serverless container service (AWS ECS with Fargate), which could work as the build agent which host the build and testing process within the CD pipeline. According to our experiments and analysis, the feature of serverless computing, combined with the beneficial brings by Docker container, could improve the parallel execution performance and save cost. The high performance in parallel pipeline execution could be helpful in several aspects: First, it improves the performance of continuous delivery for the microservices project as we stated in \ref{micros}. Second, the serverless container service reduces the effort when setting up the build agents. Lastly, our experiments also show that Pay-per-use model makes Fargate cheap to use for hosting the build agent.
\par
The second improvement is from the serverless function (AWS Lambda) could be used within the monitoring part of the toolchain. The event-driving computing model of serverless functional is perfect for processing with the logs, alarms and realtime performance metrics within monitoring. To a software team that sets up DevOps toolchain, they can use AWS Lambda to extend the monitoring system, such as customizing metrics and sending notification. In addition to these, AWS Lambda can help the software team to set up an AWS-event-driven workflow that triggers Jenkins job when changes happen in AWS services. [An example could enable Jenkins to build starts when new data is uploaded to S3 bucket.] better remove
\par
The third improvement is from the serverless managed tools (AWS CloudWatch e.t.). Those tools could be used directly as a component in the toolchain. For example, A software team can use CloudWatch directly as the monitoring solution in the toolchain. This type of tools frees the development team from developing and maintaining their monitoring solution. Suppose these tools are provided by the cloud provider that deploys the DevOps toolchain. In that case, such as monitoring AWS deployed DevOps toolchain with CloudWatch, the good integration between the tool and the cloud may provide more detailed monitoring to the cloud infrastructure than any self-built third-party tool.
\par
However, there are still limitations when using serverless computing services. One most significant limitation is that the Fargate does not support runs privileged container. This was largely limiting what we operation can we do within the running Docker build agent; for example, we cannot build a Docker image for the project that is being built in the build agent. Moreover, the performance of the build agent runs in the serverless computing service is lower. This is due to the cold-start problem of serverless computing. Also, the average time of build stage of the same software project is longer in the serverless build agent (92s) than in EC2 based build agent(65s). This difference shows that the serverless computing engine has lower performance when it runs the automated build and automated testing within a continuous delivery pipeline.
\paragraph{RQ2:} \textit{How does the newly emerged integrated toolchain in Amazon Web Services compare with the traditional non-integrated toolchain?}
\medskip
\par
Compared with non-integrated toolchain, the integrated toolchain stands out with lower development difficulty, better integration with other AWS services, less time and effort to develop and better customer support. However the lower performance and higher cost makes it not always a better solution to the software development team.
\par
% The integrated toolchain is a new type of DevOps toolchain that delivered as a single application. It is usually composed of serverless tools from a single vendor (in our case: AWS) but could also integrate with a limited range of third-party tools depends on the vendor. 
The first advantage for integrated toolchain is it is an out-of-box solution that needs much less time and effort to develop and setup. In contrast, building the non-integrated toolchain needs many works, namely, tool selection, cloud infrastructure design and management, manual configuration and even software development. Besides, an integrated toolchain provides a better sight on the DevOps toolchain as a whole. This is easier to achieve because the integrated toolchain is delivered as a single platform. In our experiment on performance, we find that the integrated toolchain is faster in provisioning the computing resource for continuous delivery. We believe due to the computing resources are also managed by AWS, it easier for AWS to optimize the provisioning process.
\par
However, we also find under the same hardware configuration, and it takes the integrated toolchain longer time to run the automated build and automated testing. The gap of the runtime between integrated and non-integrated toolchain increased with the number of parallel executing jobs.
Our research shows this is because the AWS is imitating the maximum hardware resources that we could use. The restriction shows the first disadvantage of the integrated toolchain -- the development team does not have full control on the underlying hardware. A similar problem with integrated toolchain is that the team also has low visibility on the status of the underlying hardware, which brought some challenges in our implementation. Moreover, our analysis shows that the cost of the integrated toolchain is higher in the non-integrated toolchain. The high cost is because, first, not possible to manly use open-source solution, second, the AWS charges much more on computing resource when using it from it's DevOps toolchain.
Furthermore, the integrated toolchains limit the tool selection, which on the one hand reduce the time needed for tool selection, on the other hand, it limiting the use of some special tools that are not supported by the toolchain.
\par
Although the AWS integrated toolchain has a higher cost per running hour, the higher price also means better services and less time and effort to develop and maintain the toolchain. The team also need to consider if they have specific tools they need to use that cannot be integrated with AWS toolchain. In the performance aspect, the delivery speed of AWS toolchain could vary dramatically depends on the service number in the microservices and the project complexity. Generally, it is slower compared with Jenkins build agent with the same hardware consideration. The team need to consider if the speed is a sensitive factor in their business and if the project needs parallel pipeline execution.
\section{Future Work}
Based on our current implementation and findings, we propose following further works could be done.
\begin{itemize}
    \item Due to the tight schedule in finishing the thesis, our case project is rather simple, and the test cased are rather small.  With the increasing popularity of microservices architecture, it is interesting if we extend the case project to a microservices. Such an extension could better simulate the software project in real-life development. We can further includes more type of software projects into the case projects, which could strength our conclusion by show that the conclusion is applicable to all kinds of the software projects.
    \item Current our experiment only simulate the DevOps workflow of a simple project. However, the delivery frequency, the size and the development behaviour of a software team could also affect the toolchain's performance. It could be interesting if we could test the toolchains in a real-life software development team to see how does them performs. We can also get a more user-experience related result by interviewing the team members about the experience and difficulties when develop and use the toolchains.
\end{itemize}

