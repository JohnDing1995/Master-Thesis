\chapter*{Preface}
\addcontentsline{toc}{chapter}{Preface}
Helping customers transform their software development practices to DevOps is one of the main business activities of Eficode, and DevOps toolchain is an essential part of the transformation process.
On the other hand, as an advanced partner of multiple cloud providers, Eficode is interested in what cloud technologies could bring to the DevOps toolchain. In this thesis project, I focus on the AWS serverless platform\footnote{https://aws.amazon.com/serverless/} in Amazon Web Services and discuss what change, especially benefit can it bring to a DevOps toolchain.

The process of carrying out this project is not an easy task; setting up the cloud infrastructure requires an enormous number of operational and configuration tasks. The vast but unregulated plugin eco-system of Jenkins also leads to problems like lack of plugin documentation, dependency hell and an unstable plugin such as plugin for using ECS as Jenkins agent. There does not exist that much previous research about the serverless within DevOps toolchain. Moreover, as a student with a software development background, the lack of prior experiences in the related field means much study is needed before I can start the project. All these tedious and unexpected tasks above, plus the tight thesis schedule did make me frustrated in the middle phase of this project. Despite all of these, I managed to achieve the defined goal by answering all the research questions and implement the demo. Through this project, I familiarized with different exciting tools for cloud and DevOps; it opens up a whole new area for me and will help me with my future career in Eficode. I hope the result will give Eficode more insight on the capability that AWS serverless platform could bring for the DevOps toolchain.
\vspace{1\baselineskip}

\noindent
Now I'm at the end of this two year's journey, and I feel grateful that I made this life-changing decision to come and study in Europe. I have to say, it was not an easy decision to quit the ongoing research master's study back in Shanghai, China and start all over again in a brand new environment. The two year's study was a journey full of struggle -- in both financial and study. However, what it brings is more than what I expected: international experiences by EIT Digital, great friends from different countries, two intern/work experiences, and precious knowledge that combines business and technologies. Most importantly, I could study the topic that I'm interested in and start a career within this field, which is what I was not able to do back in my previous research master's study.

I sincerely thank my supervisor Mikko Drocan from Eficode, Prof.dr.ir. D.H.J. Epema from TU Delft for their guidance and support in writing during this unprecedented time. Thanks to Eficode, for giving me this precious opportunity and sponsoring this thesis project. I also would like to give my gratitude to EIT Digital master school for the scholarship, which made it financially possible for me to finish the two-year's master's study. Lastly, special thanks to Tatu Kairi and Nils Haglund from thesis support team in Eficode for giving the weekly support on my thesis, and Eficode IT for providing the AWS cloud environment.
\vspace{1\baselineskip}


\noindent
Ruiyang Ding

\vspace{1\baselineskip}

\noindent
Helsinki, Finland

\noindent
\today
