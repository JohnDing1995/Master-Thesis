%De aankondiging bevat de spreker, titel, plaats, datum en tijd, samenstelling van de afstudeercommissie en een korte samenvatting (maximaal 25 regels).
\thispagestyle{empty}

\noindent \textbf{Author}\\
\begin{tabular}{l}
Ruiyang Ding\\
\\
\end{tabular}\\
\noindent \textbf{Title}\\
\begin{tabular}{l}
A Cloud-Based DevOps Toolchain for Efficient Software Development\\
\\
\end{tabular}\\
\noindent \textbf{MSc presentation}\\
\begin{tabular}{l}
% <MM> DD, YYYY (like \today)
31th August 2020\\
\\
\end{tabular}

\vspace{1.1cm}

\noindent \textbf{Graduation Committee}\\
\begin{tabular}{ll}
Prof dr. ir. D. H. J. Epema (chair) & Delft University of Technology \\
% The order of listing the names: Graduation prof, supervisor(s), others ordered by title + alphabetical
%examples:
Mikko Drocan & Eficode Oy \\
Dr. J.S. Rellermeyer & Delft University of Technology \\
Prof. dr. M.M. Specht & Delft University of Technology \\
\end{tabular}

\begin{abstract} %de abstract bevat alleen een korte samenvatting van de inhoud van het onderzoek
    In the traditional software development life cycle, development and operation are divided into different departments. The conflict between departments and, besides, the lack of automation usually leads to low software development efficiency and slow software delivery. Thus, the concept of DevOps is introduced which combines different departments and automates the process to make software delivery faster and easier. The DevOps toolchain is one important component for adapting DevOps. On the other hand, the adaptation of cloud technology, especially serverless computing makes it tempting for us to investigate what benefit serverless computing brings to the DevOps toolchain. In the first research question we examine the benefits that AWS serverless platforms bring to DevOps toolchain. To answer this research question, we (1) develop a DevOps toolchain hosted in Amazon Web Services (AWS), in addition, using the serverless computing service; (2) Examine what does each serverless computing service brings to the DevOps toolchain, examine how does the performance of the DevOps toolchain changes with or without using serverless computing service. 
    \par
    Our research shows that a part of the serverless computing service could reduce the cost, operation effort, and improves performance with the help of enabling parallel execution. The experiment shows that, in contrast to a toolchain hosted in a traditional cloud server vs the toolchain that was developed by us using serverless computing service could reduce the total runtime of parallel execution up to 65\%. In the second research question, we focus on the integrated toolchain build with AWS DevOps tools from AWS serverless platform. We (3) build a demo integrated DevOps toolchain with AWS DevOps tools and (4) compare the integrated toolchain with the non-integrated toolchain we build in (1). We find out that the integrated toolchain significantly reduces the development time by providing an out-of-box solution for DevOps toolchain. In addition, the better integration with underlying cloud infrastructure provides more functionality such as global monitoring and blue/green deployment. However, we also find out from the experiment that the performance of the integrated toolchain is lower due to the limitation of resources which also come with a high cost.
% Remove for now
    %In the traditional software development life cycle, development and operation are divided into different departments. The conflict between departments and, besides, the lack of automation usually leads to low software development efficiency and slow software delivery. Thus, the concept of DevOps which combines different departments and automate the process emerges. There are various DevOps platforms provided by different vendors To help the companies migrate to DevOps. To help the company select the suitable DevOps platform, it's important to analyse the needs of the company's side. The thesis project investigates what is the demand from the company on DevOps platform. This is being done by [TODO: How to get this, literature survey?]. Besides, the project compares main DevOps platforms which are being used in the projects of Eficode. The first part of the comparison is from the quantitative perspective. We compare the performance and quality of the software delivery pipeline of tested platforms by deploying a sample [TODO: What kind of application] application through the same continuous delivery model deployed on the different DevOps platforms. Furthermore, we also compared the cost of the different platform in the same testing. The second part of the comparison is from the functionality perspective. In this part, whether these platforms satisfy the need from the team which adopting DevOps is being analysed.  
\end{abstract}

\clearpage

