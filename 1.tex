\chapter{Introduction}
\label{chp:introduction}
The Agile Manifesto \cite{beck2001manifesto} drafted by Kent Beck etc. in 2001 created the Agile software development method. Since then, this new software development method has draw attention to the industry and more and more companies started to apply Agile in the production.
The Agile method advocates the shorter development iteration, continuous development of software and continuous delivery of the software to the customer. The goal \cite{beck2001manifesto} of Agile is to satisfies customer with early and continuous delivery of the software.
The Agile, which aims at the improvement of the process within the software development team and the communication between the development team and costumers \cite{miglierina2014application} do makes the software development faster. However, it doesn't emphasis the cooperation and communication between the development team and other teams. In real life, the conflict and lack of communication between the development team and operation teams usually become the barrier for shortening the delivery time of the software project.
\par
Thus, in answering to how to solve the gaps and flaws when applying Agile into the real-life software development, the concept of DevOps emerged. The term "DevOps" is created by Patrick Debois in 2009 \cite{kim2016devops}, after he saw the presentation "10 deployments per day" by John Allspaw and Paul Hammond. While Agile fills the gap between software development and business requirement from the customer, the DevOps eliminates the gap between the development team and the operation team \cite{WhatisaD20:online}. By eliminates the barrier we mentioned in the last paragraph, DevOps further fasten software delivery. In conclusion, DevOps means a combination of practices and culture which aims to combine separate departments(software development, quality assurance and the operation and others) in the same team, in order fasten the software delivery, maximizing delivered without risking high software quality \cite{DevOpsWi87:online}\cite{ebert2016devops}.
\par
In software engineering, the toolchain is a set of tools which combined for performing a specific objective. DevOps toolchain is the integration between tools that specialised in different aspect of the DevOps ecosystem, which support and coordinate the DevOps practices. The DevOps toolchain could assistant business in creating and maintain an efficient software delivery pipeline, simplify the task and further achieve DevOps \cite{DevOpsto7:online}. On the other hand, DevOps strongly rely on tools. There are specialised tools exist for helping teams adopt different DevOps practices \cite{zhu2016devops}.
\par
At the same period that the tools for DevOps emerged and developed, the cloud technologies also developed rapidly. This leads to the emigrations of Serverless Computing. 
The Serverless Computing is a new cloud computing model which all user to build and run application on the cloud, without thinking about the servers \cite{Serverle81:online}. It also allow developers to build application with less overhead \cite{Serverle81:online} and more flexibility by eliminates infrastructure management tasks \cite{Serverle73:online}.
With serverless computing technologies, many new cloud technologies emerged, which gives developers an alternative way than traditional cloud servers or cloud virtual machines. For examples: Functional computing allows the application to be divided by functions and designed under event-driving paradigm with out managing the hardware infrastructures. The on-demand nature of the serverless computing cloud could be used to deploy certain component of a DevOps toolchain to ease the implementation difficulties and reduce the cost. Managed scalable container services in the cloud enable the user to run the container-based application directly on cloud, which help the toolchain become more scalable. DevOps tools as a service \cite{DevOpsas45:online} allow the cloud provider deliver a DevOps tool directly on it's cloud platform.
\par
Helping the customer do the DevOps transformation is one of the main business activities of Eficode, the company which I'm writing my thesis. This is done by the developing and deployment DevOps toolchain for costumers. As mentioned in the last paragraph, the new changes brought by cloud may further improve the performance and lower the cost of DevOps toolchain development -- both in money and time. As part of thesis work at Eficode, We will investigate how could serverless computing could help improve the DevOps toolchain. 
\section{Problem Statement}
% Don't refer RQs here
As per last paragraph, serverless computing gives developers alternative ways to deploy DevOps tools with the new cloud technologies other than traditional cloud virtual machines. There are several cloud providers that utilise the serverless computing. Among them, Amazon Web Services(AWS)\footnote{https://aws.amazon.com/} has the largest market share is the first cloud provide which provides the serverless computing services. According to the report from Gartner \cite{GartnerS47:online}, the market share of AWS was 47.8\% in year 2018 which makes it the largest cloud provider in the world.
\par
Nowadays, the serverless computing services in AWS has already been expanded to a set of fully managed services called "AWS serverless platform" \footnote{https://aws.amazon.com/serverless/}. This platform include new AWS cloud products that leverage the serverless computing technologies. These products includes for instance, AMS Lambda\footnote{https://aws.amazon.com/lambda/} for function computing, AWS Fargate\footnote{https://aws.amazon.com/fargate/} for managed container services and AWS CodePipeline provided a managed continuous delivery pipeline as service\footnote{https://aws.amazon.com/codepipeline/} etc. 
AWS also gains the most popularity among the developers that using serverless technologies. The most recent survey report \cite{cncf2020} from Cloud Native Computing Foundation (CNCF) shows that 51\% of serverless users are using AWS Lambda, while 68\% of developers who are not using Kubernetes are using AWS ECS to hosting their containers.
As the Advanced AWS partner, AWS is being used as the main cloud providers in the customer projects by Eficode. And the company is keep looking for ways to leverage serverless computing services in AWS in order to benefit the DevOps toolchains it builds for costumers.
\par
The second area we'd like to investigate in the project is related to the integrated toolchain which is powered by the tool-as-a-services in AWS.
The integrated DevOps toolchain is delivered as a single platform that allows development teams to start using DevOps toolchain without the pain of having to choose, integrate, learn, and maintain a multitude of tools. While the traditional non-integrated toolchain is to have individual tools which are stand-alone and from different companies.
\par
This newly emerged type of toolchain is one of the new changes that cloud technologies bring, but it also leaves a question to the development team who trying to build DevOps toolchain on AWS: which kind of toolchain should they select? Should they stick on the previous non-integrated toolchain or embracing the integrated one? The integrated DevOps toolchain provides an out-of-box integrated solution for the whole DevOps lifecycle, which is tempting, but apart from the advertisement from the vendors of these "DevOps" platforms, It is still unclear if it is better than the traditional standalone toolchain. 
% So another part of our research is a case study which compares these two kinds of toolchains on different metrics. This could give the software development team a better knowledge on which kind of toolchain suites them the best.
\par
Based on the above, the research questions could be summarised as below:
\begin{enumerate}
    \item \textbf{RQ1:} How serverless computing services in Amazon Web Services helps the DevOps toolchain?
    \item \textbf{RQ2:} How does the newly emerged integrated toolchain compared with the stand-alone toolchain in Amazon Web Services?
\end{enumerate}
% DevOps strongly rely on tools. There are specialised tools exist for helping teams adopt different DevOps practices\cite{zhu2016devops}. There are different categories of tools used for different parts of the DevOps practice. For example, Jenkins for continuous integration, Ansible for process automation, sonarqube for code analysis, Slack for team communications etc. It could be interesting if we could investigate what kind of tools does it exist, and how those tools could help teams in DevOps different practice. The research could include how these tools could combine as well. 
% \par 
% Typically, the DevOps platform formulates the DevOps life cycle into the following stages:\cite{DevOps1016:online}
% \begin{itemize}
%     \item Build automation and continuous integration
%     \item Test and verification
%     \item Deployment
%     \item Operation and monitoring 
% \end{itemize}
% Move below to CH3
% \par
% The deployment of toolchain could be on-premise, which means the software development team does the deployment and integration between the tools by themselves. This could: first, provides the maximum freedom for the team to choose the tools, and secondly, allow team customise the toolchain according to their needs. However, the downside is that the need for extra time for deployment and maintenance. Besides, the cost is hard to calculate and control. 
\section{Research Method}
To answer the RQ 1 we will first build a DevOps toolchain with the popular tools used in the industry. The toolchain will be deployed on Amazon Web Services (AWS) which is the cloud services used by Eficode. 
So in the process of developing and deploy the toolchain, we will answer the RQ by research how could serverless computing in AWS benefit our toolchain. We will first conduct a literature review on new cloud technologies in chapter 3, in which we also introduce the implementation of these technologies by different cloud vendors. In this Chapter, we will also discuses which cloud technologies can benefit the DevOps toolchain. In the next step (chapter 4), we will design and conduct experiments which evaluating the benefit that each cloud technologies we researched in chapter 3 can bring. 
\par
The evaluation will be done by comparing the metrics measured from the toolchain with and without using certain cloud technologies from AWS. The metrics cover different perspectives includes cost, performance and development difficulties. We will also have the demo implementations which shows the answer to this research question.
\par
To answer RQ2, The standalone toolchain will be used to compare with the DevOps toolchain build by the DevOps tools provided by AWS as a services. We will conduct a case study on a comparison between an AWS based traditional toolchain and the out-of-box integrated DevOps toolchains also provided by AWS as services. The reason that we keep the comparison scope within AWS is that both 2 toolchain will be runs on the same hardware setup provided by AWS, this could eliminates the errors caused by the difference between vendors and focuses on the difference between toolchains.
\par
In the comparison, we will simulate the same DevOps lifecycle of a demo Spring Boot web app on both toolchains. We will again measure the metrics in these 2 toolchains, the process will be similar to what we will do on RQ1.
For software development teams, it could provide better insights on how to select the DevOps toolchains.
% The perspective of comparison between these toolchains will include:
% \begin{itemize}
%     \item Development time: The time spend for implements the toolchain and set up the whole DevOps pipeline.
%     \item Cost structure: The total cost for using the toolchain. For self-built toolchain, it will also include the cost decomposition (for different tools).
%     \item Flexibility: How much freedom can you add/change tools in the toolchain.
%     \item Scalability: How easy to scale the toolchain for the larger project.
%     \item Performance: The performance of the Continues Delivery pipeline, for the same task, how long will it take for the whole process?
% \end{itemize}
\section{Thesis Structure and Main Contributions}
In Chapter 2, we will introduce concepts within the scope of DevOps. We will also include the concepts in cloud computing which is related to our research. Chapter 3 is focusing on a survey on serverless computing technologies which the DevOps toolchain could make use of. Chapter 4 focuses on the designed and the implementation of our DevOps toolchains. Chapter 5 focuses on the experiments and evaluations, which show how does the serverless computing services introduced in CH3 could benefit DevOps toolchain, and how these 2 kinds of toolchains we mentioned earlier compared with each other. We will finally summarise our research and answer the research questions in Chapter 6.
\par
The main contributions of this paper are:
\begin{itemize}
    \item We provide a study on how could the DevOps tools leverage the cloud services to reduced development/deployment difficulties, lower the cost and improving the performance. This part of research could help the software team which is going to employ DevOps understand the practices needed. Besides, the research gives them a clearer scope of the tools needed for implementing the practices.
    \item We give the overview of 2 different types of DevOps toolchain. We also implement demo prototypes for each type of toolchain and conduct experiments with these prototypes. The experiment result shows a comparison between different toolchains. It could help the team understand which toolchain cloud be selected based on the needs.
\end{itemize}