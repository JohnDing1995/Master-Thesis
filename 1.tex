\chapter{Introduction}
\label{chp:introduction}
\justify{
The Agile Manifesto \cite{beck2001manifesto} drafted by Kent Beck et al. in 2001, created the Agile software development method. Since then, this software development method has drawn attention to the industry. Agile has become a leading standard for the software development industry, with multiple further enhancements aiming to tackle certain business-specific challenges.
The Agile method advocates the shorter development iteration, continuous development of software and continuous delivery of the software to the customer. The goal of Agile is to satisfy the customer with early and continuous delivery of the software \cite{beck2001manifesto}.
The Agile, which aims at the improvement of the process within the software development team and the communication between the development team and customers \cite{miglierina2014application} improves software development and makes it more iterative and thus faster. However, it does not emphasise the cooperation and communication between the development team and other teams. In real life, the conflict and lack of communication between the development team and operation teams usually becomes the barrier for efficient development of a software project \cite{jabbari2016devops}.}
\par
Hence, in answer to how to solve the gaps and flaws when applying Agile into real-life software development, the concept of DevOps emerged. The term "DevOps" is created by Patrick Debois in 2009 \cite{kim2016devops}, after he saw the presentation "10 deployments per day" by John Allspaw and Paul Hammond. While Agile fills the gap between software development and business requirement from the customer, the DevOps eliminates the gap between the development team and the operation team \cite{WhatisaD20:online}. By eliminates the barrier we mentioned in the last paragraph, DevOps further enhances and smoothers software delivery. In conclusion, DevOps means a combination of practices and culture which aim to combine separate departments(software development, quality assurance and the operation and others) in the same team, in order fasten the software delivery, maximising delivered without risking high software quality \cite{DevOpsWi87:online}\cite{ebert2016devops}.
\par
In software engineering, the toolchain is a set of tools which are integrated for performing a specific objective. DevOps toolchain is the integration between tools that specialised in different aspects of the DevOps ecosystem, which support and coordinate the DevOps practices. The DevOps toolchain helps organisations in creating and maintain an efficient software delivery pipeline, automate the development process \cite{DevOpsto7:online} which are the keys aspects in DevOps. On the other hand, DevOps relies strongly on tools. There exist specialised tools which help teams adopt different DevOps practices \cite{zhu2016devops}. 
\par
Traditionally, a DevOps toolchain was to have individual tools which were stand-alone and from different vendors. The tools were usually on-premise. However, such toolchains can also be deployed on cloud virtual machines. In this report, we define this type of toolchain as \textbf{non-integrated toolchain}.
\par
At the same period that the tools for DevOps emerged and developed, the cloud technologies also developed rapidly. This led to the emigrations of Serverless Computing. 
The Serverless Computing is a modern cloud computing model in which everything is built and executed in the applications running in the cloud environments without thinking about physical servers \cite{Serverle81:online}. It also allows developers to build the application with less overhead \cite{Serverle81:online} and more flexibility by eliminating infrastructure management tasks \cite{Serverle73:online}.
With serverless computing technologies, many new cloud technologies emerged, which gives developers an alternative way from traditional cloud servers or cloud virtual machines. For example, Functional computing allows the application to be divided by functions and designed under the event-driving paradigm without managing the hardware infrastructures. The on-demand nature of the serverless computing could be used to deploy event-based components of a DevOps toolchain, such as post-deploy testing and logging.
Managed scalable container services in the cloud enable the user to run the container-based application directly on the cloud, which allows the toolchain scalability. DevOps tools as a service \cite{DevOpsas45:online} allow the cloud provider to deliver DevOps tools directly on its cloud platform.
\par
Helping the customer in their DevOps transformation is one of the main business activities of Eficode, the company which we are writing our thesis. The transformation is enabled, for example, by defining, developing and maintaining a DevOps toolchain at the customer. As mentioned in the last paragraph, the new changes brought by cloud may further improve the performance and lower the cost of DevOps toolchain development -- both money and time. As part of our thesis work at Eficode, we will investigate how serverless computing enhances the DevOps toolchain,
\section{Problem Statement}
% Do not refer RQs here
As per the last paragraph, serverless computing could bring enhancements to the DevOps toolchain. Currently, there are several cloud providers that providers cloud services using serverless computing technologies. 
Among them, Amazon Web Services (AWS)\footnote{https://aws.amazon.com/} has the largest market share and is the first cloud provider which provides serverless computing services. According to the report from Gartner \cite{GartnerS47:online}, the market share of AWS was 47.8\% in the year 2018, which makes it the largest cloud provider in the world.
\par
Nowadays, the serverless computing services in AWS has already been expanded to a set of fully managed services called "AWS serverless platform" \footnote{https://aws.amazon.com/serverless/}. This platform includes new AWS cloud products that leverage the serverless computing technologies. These products include, for instance, AMS Lambda\footnote{https://aws.amazon.com/lambda/} for function computing and AWS Fargate\footnote{https://aws.amazon.com/fargate/} for managed container services.
AWS also gains the most popularity among the developers that use serverless technologies. The most recent survey report \cite{cncf2020} from Cloud Native Computing Foundation (CNCF) shows that 51\% of serverless users are using AWS Lambda, while 68\% of developers who are not using Kubernetes are using AWS ECS to hosting their containers.
As the Advanced AWS partner, AWS is being used as the main cloud providers in the customer projects by Eficode. Furthermore, the company keeps looking for ways to leverage serverless computing services in AWS to produce cost-efficient solutions for the customers.
\par
However, despite the serverless computing is being extensively used, and an enormous number of research papers about the use-cases or benefits of serverless in data analysis \cite{8457831}, for container-based microservices \cite{perez2018serverless}, or for IoT applications \cite{nastic2017serverless} \cite{glikson2017deviceless}, the benefits of serverless within DevOps has not yet been discussed. There is a paper \cite{ivanov2018implementation} and a book \cite{bangera2018devops} about DevOps toolchain for serverless applications. Nevertheless, there is still a lack of research on how serverless helps DevOps toolchain itself. Thus, our first research question is to fill the gap by answering this question. 
\par
The second area we need to investigate in our project is the integrated DevOps toolchain that is powered by serverless DevOps tooling in AWS. 
\par
The \textbf{integrated DevOps toolchain} is delivered as a cloud-based single platform that allows development teams to start using DevOps toolchain without the challenge of having to choose, integrate, learn, and maintain a multitude of tools. In other words, the cloud based-integrated DevOps toolchain is to offer DevOps toolchain as a service.
In AWS, this is offered by AWS CodePipeline (as the platform) and Several serverless tools that integrated with CodePipeline.
\par
This integrated toolchain is one of the new changes that serverless computing brings, but it also leaves a question to the development team who is trying to build DevOps toolchain in AWS: which kind of toolchain should they select? Should they stick on the previous non-integrated toolchain or embracing the integrated one? The integrated DevOps toolchain provides an out-of-box integrated solution for the whole DevOps lifecycle, which is tempting. However, apart from the advertisement from the vendors of these "DevOps" platforms, we still lack third party researches about the comparisons between these two. 
\par
Based on the above, the research questions could be summarised as below: \\
\textbf{RQ1:} \textit{How can serverless computing services in Amazon Web Services enhance the DevOps toolchain?} \\
\textbf{RQ2:} \textit{How does the integrated toolchain build with AWS DevOps Tools compare with the traditional non-integrated toolchain?}
\section{Research Approach} 
\par
To answer the RQ 1, we first investigate the current serverless offering in Amazon Web Services (AWS), which is one of the cloud services mainly used in Eficode. In chapter 3, we first introduce serverless computing services that we will use in our implementation. we are also analysis the role of each service within a DevOps toolchain.
\par
To answer RQ1 and RQ2, we implement both traditional non-integrated and integrated toolchain based on the DevOps practices and tools used by Eficode. In the design and implementation of the toolchain, we focus on the following DevOps practices: Version Control, Configuration Management, Continuous Delivery and Monitoring. The goal of the implementation is: First, validate the availability of using AWS serverless computing services in the traditional non-integrated toolchain. Thus, in the process of developing and deploy the toolchain, we could already partly answer the RQ1 by answer how the serverless computing services be used in our DevOps toolchain. Second, the implementation served as the environment for experiments in Chapter 5, which could answer two RQs.
\par
To further facilitate the answer to RQ1, our next step of the study is an experiment.
The experiments are done by comparing the metrics measured from the toolchain with and without using certain serverless computing service from AWS. These metrics cover different perspectives, which including cost, performance and ease of use. 
\par
To answer RQ2, The non-integrated toolchain is used to compare with the integrated DevOps toolchain built by the AWS DevOps tools.
Besides, we conduct a study on a comparison between an AWS based traditional toolchain and this out-of-box integrated DevOps toolchain that is also provided by AWS. The reason that we keep the comparison scope within AWS is that by doing this, we make sure that hardware in both toolchains is from AWS, this could eliminate the errors caused by the hardware difference between vendors and focuses on the difference caused by toolchains themselves.
\par
In the experiment for answering RQ2, we simulate the same DevOps lifecycle of a demo Spring Boot web app on both toolchains. we again measure the metrics in these two toolchains. The process is similar to what we do on RQ1.
For software development teams, it could provide better insights on how to select the DevOps toolchains.
% The perspective of comparison between these toolchains will include:
% \begin{itemize}
%     \item Development time: The time spend for implements the toolchain and set up the whole DevOps pipeline.
%     \item Cost structure: The total cost for using the toolchain. For self-built toolchain, it will also include the cost decomposition (for different tools).
%     \item Flexibility: How much freedom can you add/change tools in the toolchain.
%     \item Scalability: How easy to scale the toolchain for the larger project.
%     \item Performance: The performance of the Continues Delivery pipeline, for the same task, how long will it take for the whole process?
% \end{itemize}
\section{Thesis Structure and Main Contributions}
In Chapter 2, we introduce concepts within the scope of DevOps. We also introduce the concepts in cloud computing which are related to our research. Chapter 3 is focusing on a survey on serverless computing technologies which the DevOps toolchain could make use of. Chapter 4 focuses on the design and the implementation of our DevOps toolchains(both non-integrated and integrated). Chapter 5 focuses on the experiments and evaluations, which show how the serverless computing services introduced in CH3 could benefit DevOps toolchain. We also compare integrated/non-integrated toolchain in CH5. we finally summarise our research and answer the research questions in Chapter 6.
\par
The main contributions of this thesis project are:
\begin{itemize}
    \item We provide a study on how could the DevOps tools leverage the cloud services to reduced development/deployment difficulties, lower the cost and improving the performance. This part of research could help the software team which is going to employ DevOps understand the practices needed. Besides, the research gives them a clearer scope of the tools needed for implementing the practices.
    \item we give the overview of 2 different types of DevOps toolchain. We also implement demo prototypes for each type of toolchain and conduct experiments with these prototypes. The experiment result shows a comparison between different toolchains. It could help the software team understand which toolchain cloud be selected based on the needs.
\end{itemize}